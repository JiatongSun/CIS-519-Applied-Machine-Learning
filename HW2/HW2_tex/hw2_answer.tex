\documentclass{article}
\usepackage[letterpaper]{geometry}
\geometry{verbose,tmargin=1in,bmargin=1in,lmargin=1in,rmargin=1in}

\usepackage[utf8]{inputenc}
\usepackage{amsmath}
\usepackage{listings}
\usepackage{graphicx}
\usepackage{enumitem}

\title{CIS 419/519: Homework 2}
\author{Jiatong Sun}
\date{}

\begin{document}
    \maketitle
    Although the solutions are entirely my own, I consulted with the following people and sources while working on this homework: $Junfan Pan$, $https://machinelearningmastery.com/understand-the-dynamics-of-learning-rate-on-deep-learning-neural-networks/$, $https://en.wikipedia.org/wiki/Learning_rate$.
    
    \section{Gradient Descent}
        \begin{enumerate}[label=\alph*.]
            \item % a
            The implication of the learning rate $\alpha_{k}$ is to control how big a step should be taken in the gradient descent direction towards the minimum, where a too small $\alpha_{k}$ may result in a long training time and a too large $\alpha_{k}$ may lead to an overshooting training process.
            
            
            \item % b
            The implications of setting $\alpha_{k}$ as a function of k is to select an adaptive learning rate based on the training process, since the best step to take can vary as the the training goes gradually towards the minimum and a preset constant $\alpha_{k}$ may not work well in the whole process.
        \end{enumerate}
        
       \section{Linear Regression [CIS 519 ONLY]}
        
        A decision tree can include oblique splits by...
        
        
        \section{Programming Exercises}
        \textbf{Features}: What features did you choose and how did you preprocess them?
        
        \noindent\textbf{Parameters}: What parameters did you use to train your best decision tree
        
        \noindent\textbf{Performance Table}: 
        \begin{center}
            \begin{tabular}{|c|c|c|}
                \hline
                Feature Set & Accuracy & Conf. Interval [519 ONLY]\\
                \hline
                DT 1 & a & b  \\
                DT 2 & a & b  \\
                DT 3 & a & b  \\
                \hline
        \end{tabular}
                \end{center}
        
        
        
        \textbf{Conclusion}: What can you conclude from your experience?
        
\end{document}