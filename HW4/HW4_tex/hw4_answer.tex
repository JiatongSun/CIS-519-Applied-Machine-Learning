\documentclass{article}
\usepackage[letterpaper]{geometry}
\geometry{verbose,tmargin=1in,bmargin=1in,lmargin=1in,rmargin=1in}

\usepackage[utf8]{inputenc}
\usepackage{amsmath}
\usepackage{listings}
\usepackage{graphicx}
\usepackage{enumitem}
\usepackage{amssymb}

\title{CIS 419/519: Homework 4}
\author{Jiatong Sun}
\date{02/29/2020}

\begin{document}
    \maketitle
    Although the solutions are entirely my own, I consulted with the following people and sources while working on this homework:\\
$https://oeis.org/wiki/List_of_LaTeX_mathematical_symbols$\\
$https://tex.stackexchange.com/questions/122778/left-brace-including-several-lines-in-eqnarray$
    
    \section{Fitting an SVM by Hand}
        \begin{enumerate}[label=\alph*.]
            \item % a
            As is given in the problem:
            \begin{equation}
            	x1=0,\quad x2=\sqrt{2}
            \end{equation}
            and
            \begin{equation}
            	\phi(x)=[1,\sqrt{2},x^2]^T
            \end{equation}
            We know that
            \begin{equation}
            	\phi(x_1)=[1,0,0]^T
            	\quad
            	\phi(x_2)=[1,2,2]^T
            \end{equation}
            Since the optimal vector $\boldsymbol{w}$ is orthogonal to the dicision to the decision boundary, it is parallel to the vector connecting $\phi(x_1)$ and $\phi(x_2)$.\\\\
            Since
            \begin{equation}
            	\phi(x_2)-\phi(x_1)=
            	[1,2,2]^T-[1,0,0]^T=
            	[0,2,2]^T
            \end{equation}
            So $[0,2,2]^T$ is a vector that is parallel to the optimal vector $\boldsymbol{w}$.
            
            \item % b
            The margin is the distance between the two points in the 3D space.
            \begin{equation}
            	margin = ||\phi(x_2)-\phi(x_1)||
            	=\sqrt{(1-1)^2+(2-0)^2+(2-0)^2}
            	=2\sqrt{2}
            \end{equation}
            
            \item % c
            From the result of a, we can assume that
            \begin{equation}
            	\boldsymbol{w} = [0,2t,2t]^T
            \end{equation}
            So
            \begin{equation}
            	||\boldsymbol{w}||
            	=\sqrt{0^2+(2t)^2+(2t)^t}
            	=\sqrt{8t^2}=2\sqrt{2}t
            \end{equation}
            According to the relationship between $||w||$ and the length of the margin, we know that
            \begin{equation}
            	d=\frac{2}{||w||}
            	=\frac{1}{\sqrt{2}t}=2\sqrt{2}
            \end{equation}
            or
            \begin{equation}
            	t=\frac{1}{4}
            \end{equation}
            So
            \begin{equation}
            	\boldsymbol{w}
            	=[0,\frac{1}{2},\frac{1}{2}]^T
            \end{equation}
            
            \item % d
            According to SVM requirement,
            \begin{equation}
            	\begin{cases}
               		y_1(\boldsymbol{w}^T\phi(x_1)+w_0)
            		\geqslant 1\\
               		y_2(\boldsymbol{w}^T\phi(x_2)+w_0)
            		\geqslant 1
            	\end{cases}
            \end{equation}
            or
            \begin{equation}
            	\begin{cases}
               		-1\times(0+w_0)
            		\geqslant 1\\
               		1\times(2+w_0)
            		\geqslant 1
            	\end{cases}
            \end{equation}
            \begin{equation}
            	-1 \leqslant w_0 \leqslant -1
            \end{equation}
            So
            \begin{equation}
            	w_0 = -1
            \end{equation}
            
            \item % e
            \begin{equation}
            	h(x)=\boldsymbol{w}^T\phi(x)+w_0
            	=\frac{x^2}{2}+\frac{\sqrt{2}x}{2}-1
            \end{equation}
        \end{enumerate}    
        
        \section{Support Vector}
        There are two possibilies:\\\\
        1. Size of maximum margin increases, if a support vector determining the shortest margin is removed.\\\\
        2. Size of maximum margin stays the same, if the removed vector is not the one determining the shortest margin.
       
\end{document}