% --------------------------------------------------------------
% This is all preamble stuff that you don't have to worry about.
% Head down to where it says "Start here"
% --------------------------------------------------------------

\documentclass[12pt]{article}

\usepackage[margin=1in]{geometry}
\usepackage{amsmath,amsthm,amssymb}
\usepackage{graphicx}
\usepackage{subcaption}
\usepackage{algorithmicx}
\usepackage{algorithm}
\usepackage{algpseudocode}
\usepackage[colorlinks,linkcolor=blue]{hyperref}
\usepackage[noabbrev]{cleveref}
\usepackage{courier}
\usepackage{listings}




\oddsidemargin 0in
\evensidemargin 0in
\textwidth 6.5in
\topmargin -0.5in
\textheight 9.0in

\newcommand{\ignore}[1]{}
\def\pp{\par\noindent}

\newcommand{\assignment}[4]{
\thispagestyle{plain}
\newpage
\setcounter{page}{1}
\noindent
\begin{center}
\framebox{ \vbox{ \hbox to 6.28in
{CIS 419/519: Applied Machine Learning \hfill #1}
\vspace{4mm}
\hbox to 6.28in
{\hspace{2.5in}\large\bf\mbox{Homework #2}}
\vspace{4mm}
\hbox to 6.28in
{{\it Handed Out: #3 \hfill Due: #4}}
}}
\end{center}
}

\makeatletter
\renewcommand{\fnum@algorithm}{\fname@algorithm}
\makeatother

\lstset{basicstyle=\footnotesize\ttfamily,breaklines=true}
\lstset{framextopmargin=50pt,frame=bottomline}


\begin{document}

\assignment{Spring 2020}{0}{January 22}{January 27}

% --------------------------------------------------------------
%                         Start here
% --------------------------------------------------------------


{\bf Name: }  INSERT YOUR NAME HERE\\

{\bf PennKey:} INSERT YOUR PENNKEY USERNAME HERE\\

{\bf PennID:} INSERT YOUR PENNID NUMBER HERE


\section{Multiple Choice \& Written Questions} \\
\begin{enumerate}
\item[1.] 
\begin{itemize}
\item [a.]   % Just put the capital letter of your answer
\item [b.]   % after to the corresponding E.g., \item[c.]  A
\end{itemize}

\item[2.] 
\begin{itemize}
\item [a.]   % Please don't change anything else about the spacing
\item [b.]   % or formatting of this answer template.
\end{itemize}

\item[3.] 
  \begin{itemize}
  \item [a.] % The reason we're having you use this format is
  \item [b.] % that it makes it much easier for us to grade.
  \end{itemize}

\item[4.] 
  \begin{itemize}
  \item[a.] % Thanks for your help!
  \item[b.] % Put your proof here and use as many lines as you'd like!
  \end{itemize}
 \vspace{80 mm}
\end{enumerate}

\section{Python Programming Questions}

Complete questions 5 and 6 in the iPython notebook.

\end{document} 